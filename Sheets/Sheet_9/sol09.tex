\documentclass[10pt]{article}

%%%%%%%%%%%%%%%%%%%%%%%%%%%%%%%%%%%%%%%%%%%%%%%%%%%%%%%%%%%%%%%%%%%%%%%%%%%%%%%
%%% packages %%%
%%%%%%%%%%%%%%%%%%%%%%%%%%%%%%%%%%%%%%%%%%%%%%%%%%%%%%%%%%%%%%%%%%%%%%%%%%%%%%%

\usepackage[english]{babel} % Choose english language
\usepackage[labelfont = bf, font = small]{caption} % Use caption package. Use bold font for caption.
\usepackage{siunitx} % Use siunitx for unit representation.
\newcommand{\RM}[1]{\MakeUppercase{\romannumeral #1{:}}}
\usepackage{graphicx}
\usepackage{tabularx}
\usepackage{float}
\usepackage{lmodern}
\usepackage{filecontents}
\usepackage{amsmath}
\usepackage{amssymb}
\usepackage[utf8]{inputenc}
\usepackage[bottom]{footmisc}
\usepackage{leftidx}
\usepackage{subcaption}
\usepackage[explicit]{titlesec}
\usepackage{booktabs}
\usepackage{multirow}
\usepackage{multicol}
\usepackage{listings}
\usepackage{enumitem}
\usepackage{pgfplots}
\usepackage{natbib}
\usepackage{xcolor}
\usepackage{url}
\usepackage{array}
\usepackage{setspace}
\usepackage{hyperref} % Referencing
\usepackage{verbatim}
\usepackage{changepage}
\usepackage[footnote, printonlyused]{acronym}
\usepackage{scrextend}
\usepackage{geometry} % Change geometry of page layout
\usepackage{rotating}
\usepackage{longtable}
\usepackage{lscape}
\usepackage{tocloft}
\usepackage{tkz-euclide}
\usepackage{listings}
\usepackage{feynmp-auto} % Create fenynman diagrams
\usepackage{tikz-feynman} % Create fenynman diagrams
\usepackage{lipsum} % For testing. insert random text

%%%%%%%%%%%%%%%%%%%%%%%%%%%%%%%%%%%%%%%%%%%%%%%%%%%%%%%%%%%%%%%%%%%%%%%%%%%%%%%
%%% new commands and environments %%%
%%%%%%%%%%%%%%%%%%%%%%%%%%%%%%%%%%%%%%%%%%%%%%%%%%%%%%%%%%%%%%%%%%%%%%%%%%%%%%%

% Create custom font
\newenvironment{myfont}{\fontfamily{put}\selectfont}{\par}

% Adapt spacing between lines
%\doublespacing

% Delete dots from toc
\renewcommand{\cftdot}{}

% Change section label to roman
\renewcommand{\thesection}{\Roman{section}}

% Customize section layout
\newcommand{\ssection}[1]{%
  \section[#1]{\centering\normalfont\scshape #1}}
\newcommand{\ssubsection}[1]{%
  \subsection[#1]{\centering\normalfont\itshape #1}}
\newcommand{\ssubsubsection}[1]{%
  \subsubsection[#1]{\centering\normalfont #1}}

% Import tikz libraries for figures
\usetikzlibrary{positioning,shadows,arrows}

% Create footnotereferencing
\makeatletter
\newcommand\footnoteref[1]{\protected@xdef\@thefnmark{\ref{#1}}\@footnotemark}
\makeatother

% Change layout of page
\hypersetup{
  colorlinks = true,
  linkbordercolor = {red},
  citebordercolor = {red},
  menubordercolor = {blue},
  urlbordercolor = {blue},
  linktoc = {page},
  pagebackref = {True},
  pdftitle = {Solution 09},
  pdfauthor = {Nils Hoyer, Maurice Morgenthaler},
  pdfcreator  = {pdflatex},
  pdfproducer = {LaTeX}
}

% Change geometry of page
\geometry{a4paper, top = 20mm, left = 20mm, right = 20mm, bottom = 15mm, headsep = 8mm, footskip = 10mm, includeheadfoot}

% Decalre uits for SIunitx
\DeclareSIUnit\femtobarn{fb^{-1}}
\DeclareSIUnit\percent{\%}

% Define colors
\definecolor{deepblue}{rgb}{0,0,0.5}
\definecolor{deepred}{rgb}{0.6,0,0}
\definecolor{deepgreen}{rgb}{0,0.6,0.2}
\definecolor{deeporange}{rgb}{0.9,0.2,0}

% Commands for further use
\newcommand{\testStat}{\tilde{q}_{\mu}}
\newcommand{\lL}[2]{\mathcal{L}\left(#1 \middle|#2 \right)}

%%%%%%%%%%%%%%%%%%%%%%%%%%%%%%%%%%%%%%%%%%%%%%%%%%%%%%%%%%%%%%%%%%%%%%%%%%%%%%%
%%% start document %%%
%%%%%%%%%%%%%%%%%%%%%%%%%%%%%%%%%%%%%%%%%%%%%%%%%%%%%%%%%%%%%%%%%%%%%%%%%%%%%%%

\begin{document}
\begin{myfont}
\lstset{language=C++,
  basicstyle=\ttfamily,
  keywordstyle=\color{blue}\ttfamily,
  stringstyle=\color{red}\ttfamily,
  commentstyle=\color{green}\ttfamily,
  morecomment=[l][\color{magenta}]{\#}
}

\begin{center}
  \begin{Large}
    \textsc{Solution for homework assignment no. 09} \\
  \end{Large}
	\vspace*{0.4cm}
    Nils Hoyer, Maurice Morgenthaler
  \vspace*{1cm}
\end{center}

\section*{Exercise 9.1}

In this exercise we have to do a simple hypothesis testing on our own.
The test statistic to use is defined by the Neyman-Pearson lemma,

\begin{equation}
  \Lambda\left(\{x\}\right) = \frac{\mathcal{L}\left(\{x\}\,|\,H_{1}\right)}{\mathcal{L}\left(\{x\}\,|\,H_{0}\right)}
\end{equation}

\noindent where $\{x\}$ stands for the data set, $H_{0}$ for the hypothesis for \textit{background-only} signal and $H_{1}$ for the \textit{background+signal} hypothesis. \\
The theory predicts the following parameters: \\

\begin{adjustwidth}{0.4cm}{0.0cm}
  The particle mass of \SI{751}{\giga\electronvolt}, \\
  a peak-width of \SI{30}{\giga\electronvolt}, \\
  an exponential background $e^{-a\cdot x}$ with $a$ = \SI{1e-3}{\per\giga\electronvolt} and \\
  an production rate of \num{3} signal events for \num{10} background events. \\
\end{adjustwidth}

\noindent First, we generate Monte Carlo data.
As said before we use an exponential function for the background and an exponential function plus a gaussian function for background and signal

\begin{align}
f_{\textrm{bkg}} = A_{\textrm{bkg}} \cdot \textrm{exp}(- a \cdot x) \\
f_{\textrm{sig}} = A_{\textrm{sig}} \cdot \textrm{exp}\left(\frac{(x - \mu)^{2}}{\sigma}\right)
\end{align}

\noindent with the amplitudes $A_{\textrm{bkg}}$ and $A_{\textrm{sig}}$. \\
We generate \num{1000} times \num{1e5} datapoints for the \textit{background-only} and the \textit{background+signal} hypothesis.
You can find a visualisation of the first iteration of datapoints in figure \ref{fig:ex9_iter}.

\begin{figure}[H]
  \centering
  \includegraphics[width = 0.6\textwidth]{./exercise09_MCdata.png}
  \caption{Visualisation of datapoints for the \textit{background-only} (blue, dashed) and \textit{background+signal} (orange, filled) hypothesis.
  Each histogram contains \num{1e5} datapoints.}
  \label{fig:ex9_iter}
\end{figure}

\noindent Next, we calculate the test statistic $\Lambda$ for every iteration.
This is done via

\begin{equation}
\Lambda = \frac{\sum\limits_{i = 0}^{1000} \textrm{ln}\left(A_{\textrm{bkg}}\right) - a\cdot x_{i} + \textrm{ln}\left(A_{\textrm{sig}}\right) - \frac{(x_{i} - \mu)^{2}}{\sigma}}{\sum\limits_{i = 0}^{1000} \textrm{ln}\left(A_{\textrm{bkg}}\right) - a\cdot x_{i}}.
\end{equation}

\noindent The histograms for $H_{0}$ and $H_{1}$ can be found in figure \ref{fig:ex9_lambda}.

\begin{figure}[H]
  \centering
  \includegraphics[width = 0.6\textwidth]{./exercise09_Lambda.png}
  \caption{Visualisation distribution of $\Lambda$.}
  \label{fig:ex9_lambda}
\end{figure}

\end{myfont}
\end{document}
