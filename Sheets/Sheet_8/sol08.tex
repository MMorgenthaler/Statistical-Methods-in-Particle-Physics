\documentclass[10pt]{article}

%%%%%%%%%%%%%%%%%%%%%%%%%%%%%%%%%%%%%%%%%%%%%%%%%%%%%%%%%%%%%%%%%%%%%%%%%%%%%%%
%%% packages %%%
%%%%%%%%%%%%%%%%%%%%%%%%%%%%%%%%%%%%%%%%%%%%%%%%%%%%%%%%%%%%%%%%%%%%%%%%%%%%%%%

\usepackage[english]{babel} % Choose english language
\usepackage[labelfont = bf, font = small]{caption} % Use caption package. Use bold font for caption.
\usepackage{siunitx} % Use siunitx for unit representation.
\newcommand{\RM}[1]{\MakeUppercase{\romannumeral #1{:}}}
\usepackage{graphicx}
\usepackage{tabularx}
\usepackage{float}
\usepackage{lmodern}
\usepackage{filecontents}
\usepackage{amsmath}
\usepackage{amssymb}
\usepackage[utf8]{inputenc}
\usepackage[bottom]{footmisc}
\usepackage{leftidx}
\usepackage{subcaption}
\usepackage[explicit]{titlesec}
\usepackage{booktabs}
\usepackage{multirow}
\usepackage{multicol}
\usepackage{listings}
\usepackage{enumitem}
\usepackage{pgfplots}
\usepackage{natbib}
\usepackage{xcolor}
\usepackage{url}
\usepackage{array}
\usepackage{setspace}
\usepackage{hyperref} % Referencing
\usepackage{verbatim}
\usepackage{changepage}
\usepackage[footnote, printonlyused]{acronym}
\usepackage{scrextend}
\usepackage{geometry} % Change geometry of page layout
\usepackage{rotating}
\usepackage{longtable}
\usepackage{lscape}
\usepackage{tocloft}
\usepackage{tkz-euclide}
\usepackage{listings}
\usepackage{feynmp-auto} % Create fenynman diagrams
\usepackage{tikz-feynman} % Create fenynman diagrams
\usepackage{lipsum} % For testing. insert random text

%%%%%%%%%%%%%%%%%%%%%%%%%%%%%%%%%%%%%%%%%%%%%%%%%%%%%%%%%%%%%%%%%%%%%%%%%%%%%%%
%%% new commands and environments %%%
%%%%%%%%%%%%%%%%%%%%%%%%%%%%%%%%%%%%%%%%%%%%%%%%%%%%%%%%%%%%%%%%%%%%%%%%%%%%%%%

% Create custom font
\newenvironment{myfont}{\fontfamily{put}\selectfont}{\par}

% Adapt spacing between lines
%\doublespacing

% Delete dots from toc
\renewcommand{\cftdot}{}

% Change section label to roman
\renewcommand{\thesection}{\Roman{section}}

% Customize section layout
\newcommand{\ssection}[1]{%
  \section[#1]{\centering\normalfont\scshape #1}}
\newcommand{\ssubsection}[1]{%
  \subsection[#1]{\centering\normalfont\itshape #1}}
\newcommand{\ssubsubsection}[1]{%
  \subsubsection[#1]{\centering\normalfont #1}}

% Import tikz libraries for figures
\usetikzlibrary{positioning,shadows,arrows}

% Create footnotereferencing
\makeatletter
\newcommand\footnoteref[1]{\protected@xdef\@thefnmark{\ref{#1}}\@footnotemark}
\makeatother

% Change layout of page
\hypersetup{
  colorlinks = true,
  linkbordercolor = {red},
  citebordercolor = {red},
  menubordercolor = {blue},
  urlbordercolor = {blue},
  linktoc = {page},
  pagebackref = {True},
  pdftitle = {Solution 08},
  pdfauthor = {Nils Hoyer, Maurice Morgenthaler},
  pdfcreator  = {pdflatex},
  pdfproducer = {LaTeX}
}

% Change geometry of page
\geometry{a4paper, top = 20mm, left = 20mm, right = 20mm, bottom = 15mm, headsep = 8mm, footskip = 10mm, includeheadfoot}

% Decalre uits for SIunitx
\DeclareSIUnit\femtobarn{fb^{-1}}
\DeclareSIUnit\percent{\%}

% Define colors
\definecolor{deepblue}{rgb}{0,0,0.5}
\definecolor{deepred}{rgb}{0.6,0,0}
\definecolor{deepgreen}{rgb}{0,0.6,0.2}
\definecolor{deeporange}{rgb}{0.9,0.2,0}

% Commands for further use
\newcommand{\testStat}{\tilde{q}_{\mu}}
\newcommand{\lL}[2]{\mathcal{L}\left(#1 \middle|#2 \right)}

%%%%%%%%%%%%%%%%%%%%%%%%%%%%%%%%%%%%%%%%%%%%%%%%%%%%%%%%%%%%%%%%%%%%%%%%%%%%%%%
%%% start document %%%
%%%%%%%%%%%%%%%%%%%%%%%%%%%%%%%%%%%%%%%%%%%%%%%%%%%%%%%%%%%%%%%%%%%%%%%%%%%%%%%

\begin{document}
\begin{myfont}
\lstset{language=C++,
  basicstyle=\ttfamily,
  keywordstyle=\color{blue}\ttfamily,
  stringstyle=\color{red}\ttfamily,
  commentstyle=\color{green}\ttfamily,
  morecomment=[l][\color{magenta}]{\#}
}

\begin{center}
  \begin{Large}
    \textsc{Solution for homework assignment no. 08} \\
  \end{Large}
	\vspace*{0.4cm}
    Nils Hoyer, Maurice Morgenthaler
  \vspace*{1cm}
\end{center}

\section*{Exercise 8.1}

Given a note on the possibility of combining results from ATLAS and CMS we have to answer the following ten questions:

\begin{enumerate}[label = \textbf{\roman*}.]
  \item How is the CL$_{\textrm{s}}$ method used for the search of the Higgs boson?

  \noindent The CL$_{s}$ is used to calculate a confidence level by which the Higgs boson is excluded or included.
  Here CL$_{s}$ is defined as

  \begin{equation}
  	\textrm{CL}_{s}(\mu) = \frac{p_{\mu}}{1 - p_{b}},	
  \end{equation}

  \noindent where $p_{\mu}$ is the p-value for the \textit{signal + background} hypothesis and $p_{b}$ the p-value for the \textit{background - only} hypothesis.
  As an example, if $\mu = 1.0$ and CL$_{s} \leq \alpha$ we would exclude the Higgs boson with a $(1 - \alpha) \cdot \textrm{CL}_{s}$ confidence level.

  \item What is the shape of a hypothetical Higgs boson signal?

  \noindent In the paper a simple model with a Gaussian-shaped signal and flat background was considered to estimate that $1\sigma_{m}$ increments leads to a loss of sensitivity of \SI{5}{\percent}. % Also used later on for signal?

  \item How is the test statistic constructed?

  \noindent The test statistic $\testStat$ was constructed as 
  
  \begin{equation}
      \testStat = -2 \cdot \textrm{ln} \left(\frac{\mathcal{L}\left(\textrm{data}|\mu, \hat{\theta}_{\mu}\right)}{\mathcal{L}\left(\textrm{data}|\hat{\mu}, \hat{\theta}\right)}\right), \quad \textrm{with} \quad 0 \leq \hat{\mu} \leq \mu.
  \end{equation}
	
	\noindent $\mathcal{L}$ is as always the Likelihood and data refers to real observations or toy datasets. 
	$\mu$ is a \textit{signal strength modifier} which is applied to the SM Higgs boson cross sections. 
	A hat over the variable signals them the be likelihood estimators. 
	Therefore $\hat{\Theta}_{\mu}$ is the estimator given $\mu$. 
	The pair $\hat{\mu}$ and $\hat{\theta}$ are together the global maximum of the Likelihood function. 
	$\hat{\mu}$ has to be bigger than zero as the signal is positive. 

  \item How is the p-value converted to the significance?

  \noindent The p-value can be converted to a significance level by using

  \begin{equation}
  	p = \int\limits_{Z}^{\infty} \textrm{d}x \cdot \frac{1}{\sqrt{2\pi}} e^{-\frac{x^{2}}{2}} = \frac{1}{2} P_{\chi^{2}_{1}}\left(Z^{2}\right)
  \end{equation}

  \noindent and adopting the convention of a 'one-sided Gaussian tail'.
  Here $P_{\chi^{2}_{1}}\left(Z^{2}\right)$ stands for the survival function of $\chi^{2}$ for one degree of freedom.

  \item Why is the look-elsewhere effect relevant and how was it estimated?

  \noindent \noindent Since one tests for the \textit{background - only} hypothesis man times as the mass of the Higgs boson is scanned one has to consider the look-elsewhere effect.
  Because one calculates the \textit{local} p-value for fixed $m_{H}$ we have to convert to the \textit{global} variables.
  The p-value therefore becomes

  \begin{equation}
  	p_{b}^{\textrm{global}} \leq \left<N_{u_{0}}\right> \cdot e^{-\frac{u - u_{0}}{2}}
  \end{equation}

  \noindent where $\left<N_{u_{0}}\right>$ is the average number of up-crossings of the likelihood ratio scan at level $u_{0}$. \\

  \noindent Eventually this procedure will give an estimate the factor by which one needs to change the \textit{local} p-value.

  \item Why does the analysis introduce nuisance parameters and how many of them are there for ATLAS and CMS?

  \noindent Nuisance parameters are introduced such that one split sources of uncertainty to completely correlated or completely uncorrelated.
  Also, multiple uncertainties might be handled by only one nuisance parameter. \\
  In the paper only one table for the nuisance parameters which are completely correlated between ATLAS and CMS is given.
  The table features 19 different parameters.

  \item Which shape do these nuisance parameters have?

  \noindent If a parameter is unconstrained it is considered to be flat. \\
  For systematic uncertainties, for which the values are also allowed to be negative, one uses a standard Gaussian $pdf$, i.e.

  \begin{equation}
  	p\left(\theta\right) = \frac{1}{\sqrt{2\pi}\sigma} \cdot e^{-\frac{(\theta - \tilde{\theta})^{2}}{2\sigma^{2}}}.
  \end{equation}

  \noindent For parameters which are not allowed to be negative, e.g. the impact parameter, one uses a log-normal $pdf$,

  \begin{equation}
  	p\left(\theta\right) = \frac{1}{\sqrt{2\pi}\textrm{ln}(\kappa)} \cdot \frac{1}{\theta} \cdot e^{-\frac{\left(\textrm{ln}\left(\frac{\theta}{\tilde{\theta}}\right)\right)^{2}}{2\left(\textrm{ln}(\kappa)\right)^{2}}},
  \end{equation}

  \noindent where $\kappa$ characterises the width. \\
  For uncertainties associated with MC events one uses the gamma distribution, i.e.

  \begin{equation}
  	p(n) = \frac{\left(\frac{n}{\alpha}\right)^{N}}{N\,!} \cdot \frac{e^{-\frac{n}{\alpha}}}{\alpha}.
  \end{equation}


  \item How is the starting point of the Higgs boson mass chosen?

  \noindent The choise of Higgs boson mass points is driven by the two decays

  \begin{align*}
  	H & \rightarrow 2\gamma \, , \quad \textrm{and} \\
  	H & \rightarrow ZZ \rightarrow 4l.
  \end{align*}

  \noindent Depending on the model of computation it was decided to start searching at values of \SI{110}{\giga\electronvolt}.

  \item Explain what figures 8, 9 and 10 represent.

  \noindent Put very nice answer here.

  \item Explain how the likelihood of equation 20 is constructed.

  \noindent The likelihood function is simply constructed by the model given in equation 2 in the paper, i.e.

  \begin{equation}
  	\mathcal{L}\left(\textrm{data}\,|\,\mu, \theta\right) = \textrm{Poisson}\left(\mu \cdot s(\theta) + b(\theta)\right) \cdot p(\tilde{\theta}\, | \, \theta)
  \end{equation}

  \noindent where $p(\tilde{\theta}\,|\,\theta)$ is the systematic error $pdf$. \\
  Of course, one has to sum over all channels which leads to a product after applying the logarithm.
\end{enumerate}


\end{myfont}
\end{document}
