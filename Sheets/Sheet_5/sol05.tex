\documentclass[10pt]{article}

%%%%%%%%%%%%%%%%%%%%%%%%%%%%%%%%%%%%%%%%%%%%%%%%%%%%%%%%%%%%%%%%%%%%%%%%%%%%%%%
%%% packages %%%
%%%%%%%%%%%%%%%%%%%%%%%%%%%%%%%%%%%%%%%%%%%%%%%%%%%%%%%%%%%%%%%%%%%%%%%%%%%%%%%

\usepackage[english]{babel} % Choose english language
\usepackage[labelfont = bf, font = small]{caption} % Use caption package. Use bold font for caption.
\usepackage{siunitx} % Use siunitx for unit representation.
\newcommand{\RM}[1]{\MakeUppercase{\romannumeral #1{:}}}
\usepackage{graphicx}
\usepackage{tabularx}
\usepackage{float}
\usepackage{lmodern}
\usepackage{filecontents}
\usepackage{amsmath}
\usepackage{amssymb}
\usepackage[utf8]{inputenc}
\usepackage[bottom]{footmisc}
\usepackage{leftidx}
\usepackage{subcaption}
\usepackage[explicit]{titlesec}
\usepackage{booktabs}
\usepackage{multirow}
\usepackage{multicol}
\usepackage{listings}
\usepackage{pgfplots}
\usepackage{natbib}
\usepackage{xcolor}
\usepackage{url}
\usepackage{array}
\usepackage{setspace}
\usepackage{hyperref} % Referencing
\usepackage{verbatim}
\usepackage{changepage}
\usepackage[footnote, printonlyused]{acronym}
\usepackage{scrextend}
\usepackage{geometry} % Change geometry of page layout
\usepackage{rotating}
\usepackage{longtable}
\usepackage{lscape}
\usepackage{tocloft}
\usepackage{tkz-euclide}
\usepackage{listings}
\usepackage{feynmp-auto} % Create fenynman diagrams
\usepackage{tikz-feynman} % Create fenynman diagrams
\usepackage{lipsum} % For testing. insert random text

%%%%%%%%%%%%%%%%%%%%%%%%%%%%%%%%%%%%%%%%%%%%%%%%%%%%%%%%%%%%%%%%%%%%%%%%%%%%%%%
%%% new commands and environments %%%
%%%%%%%%%%%%%%%%%%%%%%%%%%%%%%%%%%%%%%%%%%%%%%%%%%%%%%%%%%%%%%%%%%%%%%%%%%%%%%%

% Create custom font
\newenvironment{myfont}{\fontfamily{put}\selectfont}{\par}

% Adapt spacing between lines
\doublespacing

% Delete dots from toc
\renewcommand{\cftdot}{}

% Change section label to roman
\renewcommand{\thesection}{\Roman{section}}

% Customize section layout
\newcommand{\ssection}[1]{%
  \section[#1]{\centering\normalfont\scshape #1}}
\newcommand{\ssubsection}[1]{%
  \subsection[#1]{\centering\normalfont\itshape #1}}
\newcommand{\ssubsubsection}[1]{%
  \subsubsection[#1]{\centering\normalfont #1}}

% Import tikz libraries for figures
\usetikzlibrary{positioning,shadows,arrows}

% Create footnotereferencing
\makeatletter
\newcommand\footnoteref[1]{\protected@xdef\@thefnmark{\ref{#1}}\@footnotemark}
\makeatother

% Change layout of page
\hypersetup{
  colorlinks = true,
  linkbordercolor = {red},
  citebordercolor = {red},
  menubordercolor = {blue},
  urlbordercolor = {blue},
  linktoc = {page},
  pagebackref = {True},
  pdftitle = {Solution 05},
  pdfauthor = {Nils Hoyer, Maurice Morgenthaler},
  pdfcreator  = {pdflatex},
  pdfproducer = {LaTeX}
}

% Change geometry of page
\geometry{a4paper, top = 20mm, left = 20mm, right = 20mm, bottom = 15mm, headsep = 8mm, footskip = 10mm, includeheadfoot}

% Decalre uits for SIunitx
\DeclareSIUnit\femtobarn{fb^{-1}}
\DeclareSIUnit\percent{\%}

% Define colors
\definecolor{deepblue}{rgb}{0,0,0.5}
\definecolor{deepred}{rgb}{0.6,0,0}
\definecolor{deepgreen}{rgb}{0,0.6,0.2}
\definecolor{deeporange}{rgb}{0.9,0.2,0}

%%%%%%%%%%%%%%%%%%%%%%%%%%%%%%%%%%%%%%%%%%%%%%%%%%%%%%%%%%%%%%%%%%%%%%%%%%%%%%%
%%% start document %%%
%%%%%%%%%%%%%%%%%%%%%%%%%%%%%%%%%%%%%%%%%%%%%%%%%%%%%%%%%%%%%%%%%%%%%%%%%%%%%%%

\begin{document}
\begin{myfont}
\lstset{language=C++,
  basicstyle=\ttfamily,
  keywordstyle=\color{blue}\ttfamily,
  stringstyle=\color{red}\ttfamily,
  commentstyle=\color{green}\ttfamily,
  morecomment=[l][\color{magenta}]{\#}
}

\begin{center}
  \begin{Large}
    \textsc{Solution for homework assignment no. 05} \\
  \end{Large}
	\vspace*{0.4cm}
    Nils Hoyer, Maurice Morgenthaler
  \vspace*{1cm}
\end{center}

\section*{Exercise 5.1}

\begin{itemize}
  \item[\textbf{a)}] We are asked to find the position $\alpha$ for a given distance $\beta =$ \num{30}.
  To do this we maximize a likelihood function based on the probability $p(x;\alpha, \beta)$ which is given as

  \begin{equation}
    p(x; \alpha, \beta) = \prod_{i = 1}^{n} 
  \end{equation}
\end{itemize}



\section*{Exercise 5.2}

After importing the dataset we plotted it to confirm that we indeed have a gaussian distribution

\begin{equation}
p(x;\mu, \sigma) = \frac{1}{\sigma\sqrt{2\pi}} e^{-\frac{\left(x - \mu\right)^{2}}{2\sigma^{2}}}.
\end{equation}

\noindent The negative loglikelihood function then looks like

\begin{align*}
l(\mu, \sigma; x) & = \textrm{ln}\left(\prod_{i = 1}^{n}p(x_{i}; \mu, \sigma)\right) \\
                  & = \sum_{i = 1}^{n} \left(-\textrm{ln}\left(\sigma\sqrt{2\pi}\right) - \frac{\left(x - \mu\right)^{2}}{2\sigma^{2}}\right) \\
                  & = - \sum_{i = 1}^{n} \left(\textrm{ln}\left(\sigma\sqrt{2\pi}\right) + \frac{\left(x - \mu\right)^{2}}{2\sigma^{2}}\right)
\end{align*}

The resulting plot is illustrated in figure \ref{fig:gaussian_distribution}.
The output by the fit is given in tabke \ref{tab:ex5_2}.

\begin{figure}[H]
  \centering
  \caption{Distribution of data points given in the file 'data$\_$05.h'.
  The distribution of points seem to follow a gaussian distribution.}
  \includegraphics[width = \textwidth]{./exercise5_2.png}
  \label{fig:gaussian_distribution}
\end{figure}

\begin{longtable}{*{3}l}
\caption{Summary of all fitted parameters.}\endfirsthead
\endhead
\toprule
\textbf{Parameter} & \multicolumn{2}{c}{\textbf{Value}} \\
\cmidrule{2-3}
 & \textbf{fixed $\sigma$} & \textbf{unfixed $\sigma$} \\
\midrule
$\mu$    & \num{39.916} & \num{39.916} \\ 
$\sigma$ & \num{0.15}   & \num{0.089}  \\
\bottomrule
\label{tab:ex5_2}
\end{longtable}

\end{myfont}
\end{document}