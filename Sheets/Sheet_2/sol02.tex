\documentclass[10pt]{article}

%%%%%%%%%%%%%%%%%%%%%%%%%%%%%%%%%%%%%%%%%%%%%%%%%%%%%%%%%%%%%%%%%%%%%%%%%%%%%%%
%%% packages %%%
%%%%%%%%%%%%%%%%%%%%%%%%%%%%%%%%%%%%%%%%%%%%%%%%%%%%%%%%%%%%%%%%%%%%%%%%%%%%%%%

\usepackage[english]{babel} % Choose english language
\usepackage[labelfont = bf, font = small]{caption} % Use caption package. Use bold font for caption.
\usepackage{siunitx} % Use siunitx for unit representation.
\newcommand{\RM}[1]{\MakeUppercase{\romannumeral #1{:}}}
\usepackage{graphicx}
\usepackage{tabularx}
\usepackage{float}
\usepackage{lmodern}
\usepackage{filecontents}
\usepackage{amsmath}
\usepackage{amssymb}
\usepackage[utf8]{inputenc}
\usepackage[bottom]{footmisc}
\usepackage{leftidx}
\usepackage{subcaption}
\usepackage[explicit]{titlesec}
\usepackage{booktabs}
\usepackage{multirow}
\usepackage{multicol}
\usepackage{listings}
\usepackage{pgfplots}
\usepackage{natbib}
\usepackage{color}
\usepackage{url}
\usepackage{array}
\usepackage{setspace}
\usepackage{hyperref} % Referencing
\usepackage{verbatim}
\usepackage{changepage}
\usepackage[footnote, printonlyused]{acronym}
\usepackage{scrextend}
\usepackage{geometry} % Change geometry of page layout
\usepackage{rotating}
\usepackage{longtable}
\usepackage{lscape}
\usepackage{tocloft}
\usepackage{listings}
\usepackage{feynmp-auto} % Create fenynman diagrams
\usepackage{tikz-feynman} % Create fenynman diagrams
\usepackage{lipsum} % For testing. insert random text

%%%%%%%%%%%%%%%%%%%%%%%%%%%%%%%%%%%%%%%%%%%%%%%%%%%%%%%%%%%%%%%%%%%%%%%%%%%%%%%
%%% new commands and environments %%%
%%%%%%%%%%%%%%%%%%%%%%%%%%%%%%%%%%%%%%%%%%%%%%%%%%%%%%%%%%%%%%%%%%%%%%%%%%%%%%%

% Create custom font
\newenvironment{myfont}{\fontfamily{put}\selectfont}{\par}

% Adapt spacing between lines
\doublespacing

% Delete dots from toc
\renewcommand{\cftdot}{}

% Change section label to roman
\renewcommand{\thesection}{\Roman{section}} 

% Customize section layout
\newcommand{\ssection}[1]{%
  \section[#1]{\centering\normalfont\scshape #1}}
\newcommand{\ssubsection}[1]{%
  \subsection[#1]{\centering\normalfont\itshape #1}}  
\newcommand{\ssubsubsection}[1]{%
  \subsubsection[#1]{\centering\normalfont #1}}

% Import tikz libraries for figures
\usetikzlibrary{positioning,shadows,arrows}

% Create footnotereferencing
\makeatletter
\newcommand\footnoteref[1]{\protected@xdef\@thefnmark{\ref{#1}}\@footnotemark}
\makeatother

% Change layout of page
\hypersetup{
	colorlinks = true,
  linkbordercolor = {red},
  citebordercolor = {red},
  menubordercolor = {blue},
  urlbordercolor = {blue},
  linktoc = {page},
  pagebackref = {True},
  pdftitle = {Solution 01},
  pdfauthor = {Nils Hoyer},
  pdfcreator  = {pdflatex},
  pdfproducer = {LaTeX}
}

% Change geometry of page
\geometry{a4paper, top = 20mm, left = 20mm, right = 20mm, bottom = 15mm, headsep = 8mm, footskip = 10mm, includeheadfoot}

% Decalre uits for SIunitx
\DeclareSIUnit\femtobarn{fb^{-1}}

% Define colors
\definecolor{deepblue}{rgb}{0,0,0.5}
\definecolor{deepred}{rgb}{0.6,0,0}
\definecolor{deepgreen}{rgb}{0,0.6,0.2}
\definecolor{deeporange}{rgb}{0.9,0.2,0}

%%%%%%%%%%%%%%%%%%%%%%%%%%%%%%%%%%%%%%%%%%%%%%%%%%%%%%%%%%%%%%%%%%%%%%%%%%%%%%%
%%% start document %%%
%%%%%%%%%%%%%%%%%%%%%%%%%%%%%%%%%%%%%%%%%%%%%%%%%%%%%%%%%%%%%%%%%%%%%%%%%%%%%%%

\begin{document}
\begin{myfont}
\lstset{language=C++,
  basicstyle=\ttfamily,
  keywordstyle=\color{blue}\ttfamily,
  stringstyle=\color{red}\ttfamily,
  commentstyle=\color{green}\ttfamily,
  morecomment=[l][\color{magenta}]{\#}
}



\begin{center}
	\begin{Large}
		\textsc{Solution for homework assignment no. 02} \\
	\end{Large}
	\vspace*{0.4cm}
		Nils Hoyer, Maurice Morgenthaler
		\vspace*{1cm}
\end{center}

\section*{Exercise 2.1}

We are given a function $f$ which is defined as
\begin{equation}
f(x) =
\begin{cases}
0                           & \textrm{for}\; a \leq x \leq b \\
\frac{2(x-a)}{(b-a)(c-a)}   & \textrm{for}\; a \leq x < c \\
\frac{2(b-x)}{(b-a)(b-c)}   & \textrm{for}\; c \leq x < b \\
\end{cases}
\end{equation}

\noindent and are asked to calculate its mean, mode, median and variance.
We will consider the case where $c = 0$. \\

\noindent Mean $\mu$: $\;\mu = E[x] = \int\limits_{-\infty}^{\infty}dx \cdot xf(x)$ \\
\noindent Median $M$: $\;\frac{1}{2} = \int\limits_{-\infty}^{M}dx \cdot f(x)$ \\
\noindent Mode $m$: $\;\frac{\partial f}{\partial x}(x)|_{x = m} = 0$ \\
\noindent Varianz $\sigma^{2}$: $\;\sigma^{2} = E[(x - E[x])^{2}] = \int\limits_{-\infty}^{\infty}dx \cdot (x - \mu)^{2}f(x)$

\begin{itemize}
  \item[\textbf{a)}] For $a = -b$ the function $f(x)$ is defined as
  $$
  f(x) = 
  \begin{cases}
  0                         & \textrm{for}\; x \leq -b \; \textrm{and}\; x \geq b \\
  \frac{x+b}{b^{2}}         & \textrm{for}\; -b \leq x \leq 0 \\
  \frac{b-x}{b^{2}}         & \textrm{for}\; 0 \leq x \leq b \\
  \end{cases}
  $$
  \noindent \textbf{Mean}: Since $f(x)$ and $x$ are odd functions their product is an even function. 
  Since we integrate over a symmetric interval the mean is zero. \\

  \noindent \textbf{Median}: Because of symmerty we know that $M = 0$. \\

  \noindent \textbf{Mode}: Again, due to symmetry and because of the fact that the function is rising for $x \leq 0$ and falling for $x \geq 0$, the mode is zero. \\

  \noindent \textbf{Varianz}:
  \begin{align*}
  \sigma^{2} & = \int\limits_{-\infty}^{\infty}dx \cdot (x - \mu)^{2}f(x) = \int\limits_{-b}^{0}dx \cdot x^{2}\frac{x+b}{b^{2}} + \int\limits_{0}^{b}dx \cdot x^{2}\frac{b-x}{b^{2}} \\
             & = \frac{1}{b^{2}}\left[-\frac{b^{4}}{4} + \frac{b^{4}}{3} + \frac{b^{4}}{3} - \frac{b^{4}}{4}\right] = \frac{b^{2}}{6}
  \end{align*}

  \item[\textbf{b)}] For $a = -2b$ the function $f(x)$ is defined as
  $$
  f(x) = 
  \begin{cases}
  0                            & \textrm{for}\; x \leq -2b \; \textrm{and}\; x \geq b \\
  \frac{x+b}{3b^{2}}           & \textrm{for}\; -2b \leq x \leq 0 \\
  \frac{2}{3}\frac{b-x}{b^{2}} & \textrm{for}\; 0 \leq x \leq b \\
  \end{cases}
  $$

  \noindent \textbf{Mean}:
  \begin{align*}
  \mu & = \int\limits_{-\infty}^{\infty}dx \cdot xf(x) = \int\limits_{-2b}^{0}dx \cdot x\frac{x+2b}{3b^{2}} + \int\limits_{0}^{b}dx \cdot x\frac{2}{3}\frac{b-x}{b^{2}} \\
      & = \frac{1}{3b^{2}}\left[\frac{6b^{3}}{3} - 4b^{3} + \frac{b^{3}}{2} - \frac{b^{3}}{3}\right] = -\frac{11b}{18}
  \end{align*}


  \noindent \textbf{Median}: I suspect that the median lies between $x = -2b$ and $x = 0$.
  \begin{align*}
  \frac{1}{2}         & = \int\limits_{-\infty}^{M}dx \cdot f(x) = \int\limits_{-2b}^{M}dx \cdot \frac{x+b}{3b^{2}} \\
                      & = \frac{1}{3b^{2}} \left[\frac{M^{2}}{2} + Mb\right] \\
  \Rightarrow M_{1,2} & = -3b, b
  \end{align*}

  \noindent \textbf{Mode}: Same argument as before: Since $f_{1}$ (function defined for $x \leq 0$) is rising and $f_{2}$ is falling the mode lies at zero. \\

  %\noindent \textbf{Varianz}: 
  %\begin{align*}
  %\sigma^{2} & = \int\limits_{-\infty}^{\infty}dx \cdot (x - \mu)^{2}f(x) = \int\limits_{-2b}^{0}dx \cdot (x - \mu)^{2}\frac{x+b}{3b^{2}} + \int\limits_{0}^{b}dx \cdot (x - \mu)^{2}\frac{2(b-x)}{3b^{2}} \\
  %           & = ... \\
  %           & = \frac{1}{3b^{2}}\left[- \frac{5b^{4}}{6} + \frac{154b^{3}}{27} - \frac{238b^{2}}{27} + \left(\frac{34}{18}\right)^{2}\right]
  %\end{align*}
\end{itemize}

\end{myfont}
\end{document}