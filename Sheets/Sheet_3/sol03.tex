\documentclass[10pt]{article}

%%%%%%%%%%%%%%%%%%%%%%%%%%%%%%%%%%%%%%%%%%%%%%%%%%%%%%%%%%%%%%%%%%%%%%%%%%%%%%%
%%% packages %%%
%%%%%%%%%%%%%%%%%%%%%%%%%%%%%%%%%%%%%%%%%%%%%%%%%%%%%%%%%%%%%%%%%%%%%%%%%%%%%%%

\usepackage[english]{babel} % Choose english language
\usepackage[labelfont = bf, font = small]{caption} % Use caption package. Use bold font for caption.
\usepackage{siunitx} % Use siunitx for unit representation.
\newcommand{\RM}[1]{\MakeUppercase{\romannumeral #1{:}}}
\usepackage{graphicx}
\usepackage{tabularx}
\usepackage{float}
\usepackage{lmodern}
\usepackage{filecontents}
\usepackage{amsmath}
\usepackage{amssymb}
\usepackage[utf8]{inputenc}
\usepackage[bottom]{footmisc}
\usepackage{leftidx}
\usepackage{subcaption}
\usepackage[explicit]{titlesec}
\usepackage{booktabs}
\usepackage{multirow}
\usepackage{multicol}
\usepackage{listings}
\usepackage{pgfplots}
\usepackage{natbib}
\usepackage{xcolor}
\usepackage{url}
\usepackage{array}
\usepackage{setspace}
\usepackage{hyperref} % Referencing
\usepackage{verbatim}
\usepackage{changepage}
\usepackage[footnote, printonlyused]{acronym}
\usepackage{scrextend}
\usepackage{geometry} % Change geometry of page layout
\usepackage{rotating}
\usepackage{longtable}
\usepackage{lscape}
\usepackage{tocloft}
\usepackage{listings}
\usepackage{feynmp-auto} % Create fenynman diagrams
\usepackage{tikz-feynman} % Create fenynman diagrams
\usepackage{lipsum} % For testing. insert random text

%%%%%%%%%%%%%%%%%%%%%%%%%%%%%%%%%%%%%%%%%%%%%%%%%%%%%%%%%%%%%%%%%%%%%%%%%%%%%%%
%%% new commands and environments %%%
%%%%%%%%%%%%%%%%%%%%%%%%%%%%%%%%%%%%%%%%%%%%%%%%%%%%%%%%%%%%%%%%%%%%%%%%%%%%%%%

% Create custom font
\newenvironment{myfont}{\fontfamily{put}\selectfont}{\par}

% Adapt spacing between lines
\doublespacing

% Delete dots from toc
\renewcommand{\cftdot}{}

% Change section label to roman
\renewcommand{\thesection}{\Roman{section}}

% Customize section layout
\newcommand{\ssection}[1]{%
  \section[#1]{\centering\normalfont\scshape #1}}
\newcommand{\ssubsection}[1]{%
  \subsection[#1]{\centering\normalfont\itshape #1}}
\newcommand{\ssubsubsection}[1]{%
  \subsubsection[#1]{\centering\normalfont #1}}

% Import tikz libraries for figures
\usetikzlibrary{positioning,shadows,arrows}

% Create footnotereferencing
\makeatletter
\newcommand\footnoteref[1]{\protected@xdef\@thefnmark{\ref{#1}}\@footnotemark}
\makeatother

% Change layout of page
\hypersetup{
  colorlinks = true,
  linkbordercolor = {red},
  citebordercolor = {red},
  menubordercolor = {blue},
  urlbordercolor = {blue},
  linktoc = {page},
  pagebackref = {True},
  pdftitle = {Solution 03},
  pdfauthor = {Nils Hoyer, Maurice Morgenthaler},
  pdfcreator  = {pdflatex},
  pdfproducer = {LaTeX}
}

% Change geometry of page
\geometry{a4paper, top = 20mm, left = 20mm, right = 20mm, bottom = 15mm, headsep = 8mm, footskip = 10mm, includeheadfoot}

% Decalre uits for SIunitx
\DeclareSIUnit\femtobarn{fb^{-1}}

% Define colors
\definecolor{deepblue}{rgb}{0,0,0.5}
\definecolor{deepred}{rgb}{0.6,0,0}
\definecolor{deepgreen}{rgb}{0,0.6,0.2}
\definecolor{deeporange}{rgb}{0.9,0.2,0}

%%%%%%%%%%%%%%%%%%%%%%%%%%%%%%%%%%%%%%%%%%%%%%%%%%%%%%%%%%%%%%%%%%%%%%%%%%%%%%%
%%% start document %%%
%%%%%%%%%%%%%%%%%%%%%%%%%%%%%%%%%%%%%%%%%%%%%%%%%%%%%%%%%%%%%%%%%%%%%%%%%%%%%%%

\begin{document}
\begin{myfont}
\lstset{language=C++,
  basicstyle=\ttfamily,
  keywordstyle=\color{blue}\ttfamily,
  stringstyle=\color{red}\ttfamily,
  commentstyle=\color{green}\ttfamily,
  morecomment=[l][\color{magenta}]{\#}
}

\begin{center}
  \begin{Large}
    \textsc{Solution for homework assignment no. 03} \\
  \end{Large}
	\vspace*{0.4cm}
    Nils Hoyer, Maurice Morgenthaler
  \vspace*{1cm}
\end{center}

\section*{Exercise 3.1}

We are asked about the probability density function (PDF) of $f(z)$ where $z = x \times y$, $x$ and $y$ are independent variables and $h(x)$ and $g(y)$ are their PDF's, respectively.
The hint gives us the relation that for $n$ \underline{independent} variables we can get the PDF $f(z)$ of $z = (x_{1}, ..., x_{n})$ via

\begin{equation}
  f(z)dz := \int \; ... \; \int_{\partial S} dx_{1} ... dx_{n} \cdot g(x_{1}, ..., x_{n})
\end{equation}

\noindent where $\partial S$ is the boundary of the system of parameters. \\
We start with
$$
f(z)dz = \int\int dxdy \cdot g(x, y),
$$
\noindent such that our task is to determine $g(x, y)$. \\

\noindent From the lecture we know that $g(x, y) = g_{x}(x) \cdot g_{y}(y)$. \\
Therefore

\begin{equation}
  f(z)dz = \int\int dxdy \cdot h(x)g(y)
\end{equation}


\section*{Exercise 3.2}

We are asked to determine what the expectation value of $E[z]$ is where $z = z(x)$ and
\begin{equation}
  E[x] = \int\limits_{-\infty}^{\infty} dx \cdot xf(x).
\end{equation}

\begin{align*}
E[z(x)] & = \int\limits_{-\infty}^{\infty} dz(x) \cdot z(x)f(z(x)) \\
     & = \int\limits_{-\infty}^{\infty} dx \frac{dz}{dx} \cdot x\frac{z}{x}f\left(z(x)\right) \\
     & = \int\limits_{-\infty}^{\infty} dx \cdot x \underbrace{f\left(z(x)\right) \cdot \frac{z}{x}}_{ = f(x)} \\
     & = \int\limits_{-\infty}^{\infty} dx \cdot x f(x)
\end{align*}


\section*{Exercise 3.3}

Please note that we will only include grahpics and solutions here but not the code itself as it would take up a few pages.

\begin{itemize}
  \item[\textbf{a)}] \label{part_a}
    We are asked to read the .txt-file and plots each variable into a separate histogram as well as creating two-dimensional histograms to visualize potential correlations.
    Please find the plots in figure \ref{fig:ex_a_individual} and \ref{fig:ex_a_correlation}.

    \begin{figure}[H]
      \centering
      \caption[]{Histograms of the momentum $p$, energy $E$, normed velocity $\beta$ and mass $m$.}
      \includegraphics[width = 0.5 \textheight]{./ex_a_individual.png}
      \label{fig:ex_a_individual}
    \end{figure}

    \begin{figure}[H]
      \centering
      \caption[]{Histograms of potential correlations between the momentum $p$, energy $E$, normed velocity $\beta$ and mass $m$.}
      \includegraphics[width = 0.5 \textheight]{./ex_a_correlation.png}
      \label{fig:ex_a_correlation}
    \end{figure}

  \item[\textbf{b)}] \label{part_b}
    We are aksed to explicitly determine the mean and variance for each variable as well as obtaining the covariances for each set of pairs of variables.
    The results from the explicit determination as well as the results given by functions from ROOT are given in table \ref{tab:ex_b_results} which you can find below.

    \begin{longtable}{*{7}l}
      \caption[]{Results obtained for each variable and pair of variables. \\ Results to be added.}
      \endfirsthead
      \endhead
      \toprule
      \multirow{2}{*}{\textbf{Set of variables}} & \multicolumn{2}{c}{\textbf{Mean in [MeV]}} & \multicolumn{2}{c}{\textbf{Variance in [MeV$^{2}$]}} & \multicolumn{2}{c}{\textbf{Covariance in [MeV$^{2}$]}} \\
      \cmidrule(lr){2-3} \cmidrule(lr){4-5} \cmidrule(lr){6-7}
                       & \textbf{Code} & \textbf{ROOT} & \textbf{Code} & \textbf{ROOT} & \textbf{Code} & \textbf{ROOT} \\
      \midrule
      $p$              & \num{287.0} & \num{287.0} & \num{32.80} & \num{32.46} & $-$       & $-$       \\
      $E$              & \num{318.9} & \num{318.9} & \num{50.54} & \num{50.22} & $-$       & $-$       \\
      $\beta$          & \num{0.900} & \num{0.900} & \num{0.001} & \num{0.001} & $-$       & $-$       \\
      $m$              & \num{138.9} & \num{138.9} & \num{61.47} & \num{60.32} & $-$       & $-$       \\ \hline
      $p \times E$     & $-$       & $-$       & $-$       & $-$       & \num{36.38} & \num{35.84} \\
      $p \times \beta$ & $-$       & $-$       & $-$       & $-$       & \num{0.001} & \num{0.001} \\
      $p \times m$     & $-$       & $-$       & $-$       & $-$       & \num{15.71} & \num{15.31} \\
      $E \times \beta$ & $-$       & $-$       & $-$       & $-$       & \num{-0.029} & \num{-0.028} \\
      $E \times m$     & $-$       & $-$       & $-$       & $-$       & \num{40.87} & \num{39.98} \\
      $\beta \times m$ & $-$       & $-$       & $-$       & $-$       & \num{-0.215} & \num{0.0} \\
      \bottomrule
      \label{tab:ex_b_results}
    \end{longtable}

  \item[\textbf{c)}] \label{part_c}
    We are asked to determine and explain the correlation coefficients between the paris of variables.
    Please find the results for the coefficients in table \ref{tab:ex_c_coefficients}.

    \begin{longtable}{*{3}l}
      \caption[]{Correlation coefficients between each pair of variables. \\ }
      \endfirsthead
      \endhead
      \toprule
      \multirow{2}{*}{\textbf{Pair of variables}} & \multicolumn{2}{c}{\textbf{Correlation coefficient}} \\
      \cmidrule(lr){2-3}
       & \textbf{Code} & \textbf{ROOT} \\
      \midrule
      $p \times E$     & \num{0.893}  & \num{0.892} \\
      $p \times \beta$ & \num{0.004}  & \num{0.004} \\
      $p \times m$     & \num{0.350}  & \num{0.346} \\
      $E \times \beta$ & \num{-0.445} & \num{-0.442} \\
      $E \times m$     & \num{0.733}  & \num{0.730} \\
      $\beta \times m$ & \num{-0.935} & \num{0.934} \\
      \bottomrule
      \label{tab:ex_c_coefficients}
    \end{longtable}

    \noindent To explain these correlation coefficients we will have a look at the analytically formulae.
    For relativistic particles we have
    \begin{align}
      \gamma      & = \frac{E}{m}, \label{eq:gamma}\\
      \beta       & = \frac{p}{E}, \label{eq:beta}\\
      \gamma\beta & = \frac{p}{m} \quad \textrm{and} \label{eq:gammabeta}\\
      \gamma      & = \sqrt{\frac{1}{1 - \beta^{2}}}. \label{eq:gamma2}
    \end{align}

    \noindent Note that $\gamma\beta = \frac{\beta}{\sqrt{1-\beta^{2}}} \rightarrow \beta = \frac{\gamma}{\sqrt{1 + \gamma^{2}}}$. \\

    \noindent From equation \ref{eq:gamma} we can directly see that $E \propto m$.
    From equation \ref{eq:beta} we can see that $E \propto p$.

  \item[\textbf{d)}] \label{part_d}
    We are asked to calculate the covariance $\textrm{cov}[m, E]$ analytically with the values for $p$ and $\beta$ of part a and compare the result to the one obtained in part b.\\
   The covariance $\textrm{cov}[m, E]$ can be written as:
   \begin{equation}
   \textrm{cov}[m, E] = \frac{1}{2n^2} \sum_{i=1}^n \sum_{j=1}^n (m_i - m_j) \cdot (E_i - E_j)
   \end{equation}

   With equation \ref{eq:beta} and \ref{eq:gammabeta} we get as formula:
   \begin{equation}
   \textrm{cov}[m, E] = \frac{1}{2n^2} \sum_{i=1}^n \sum_{j=1}^n \left(\frac{p_i \cdot \sqrt{1- \beta_i^2}}{\beta_i} -  \frac{p_j \cdot \sqrt{1- \beta_j^2}}{\beta_j}\right) \cdot \left(\frac{p_i}{\beta_i} - \frac{p_j}{\beta_j} \right)
   \end{equation}
   Using the values of part a results in $\textrm{cov}[m, E] = 0.0002 $. This value is very close to $\textrm{cov}[p, \beta]$.
  

  \item[\textbf{e)}] \label{part_e}
    We are asked to calculate the uncertainity of $T = E - m$ taken the correlation between $E$ and $m$ into account. \\
    In general the uncertainty $u_y$ of a value $y$ with correlated variables $x_i$ is given by:
   \begin{equation}
    u_y = \sqrt{\sum_{i=1}^m  \left(\frac{\delta y}{\delta x_i} \cdot u_i\right)^2 + 2 \sum_{i=1}^{m-1} \sum_{k=i+1}^{m}  \left(\frac{\delta y}{\delta x_i}\right)  \left(\frac{\delta y}{\delta x_k}\right) \cdot \textrm{cov}[x_i, x_k]}
   \end{equation}
   
   For our formula it reduces to:
  \begin{align}
    u_T &= \sqrt{ (u_E)^2 + (u_m)^2 - 2 \¢dot cov[E, m]} \\
           &= \sqrt{ \textrm{std}[E]^2 +\textrm{ std}[m]^2 - 2 \cdot \textrm{cov}[E, m]} \\
           &= \sqrt{\textrm{var}[E] + \textrm{var}[m] - 2 \cdot \textrm{cov}[E, m]} 
   \end{align}
  With the values from table \ref{tab:ex_b_results} we get an uncertainty of $u_T = 5.50 MeV$. Without the correlation term the uncertainty would be $u_T^* = 10.58 MeV$ so it makes a difference of $\Delta u_T = 5.08 MeV$.
   
\end{itemize}

\end{myfont}
\end{document}
