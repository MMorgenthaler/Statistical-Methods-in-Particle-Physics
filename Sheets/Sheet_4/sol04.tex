\documentclass[10pt]{article}

%%%%%%%%%%%%%%%%%%%%%%%%%%%%%%%%%%%%%%%%%%%%%%%%%%%%%%%%%%%%%%%%%%%%%%%%%%%%%%%
%%% packages %%%
%%%%%%%%%%%%%%%%%%%%%%%%%%%%%%%%%%%%%%%%%%%%%%%%%%%%%%%%%%%%%%%%%%%%%%%%%%%%%%%

\usepackage[english]{babel} % Choose english language
\usepackage[labelfont = bf, font = small]{caption} % Use caption package. Use bold font for caption.
\usepackage{siunitx} % Use siunitx for unit representation.
\newcommand{\RM}[1]{\MakeUppercase{\romannumeral #1{:}}}
\usepackage{graphicx}
\usepackage{tabularx}
\usepackage{float}
\usepackage{lmodern}
\usepackage{filecontents}
\usepackage{amsmath}
\usepackage{amssymb}
\usepackage[utf8]{inputenc}
\usepackage[bottom]{footmisc}
\usepackage{leftidx}
\usepackage{subcaption}
\usepackage[explicit]{titlesec}
\usepackage{booktabs}
\usepackage{multirow}
\usepackage{multicol}
\usepackage{listings}
\usepackage{pgfplots}
\usepackage{natbib}
\usepackage{xcolor}
\usepackage{url}
\usepackage{array}
\usepackage{setspace}
\usepackage{hyperref} % Referencing
\usepackage{verbatim}
\usepackage{changepage}
\usepackage[footnote, printonlyused]{acronym}
\usepackage{scrextend}
\usepackage{geometry} % Change geometry of page layout
\usepackage{rotating}
\usepackage{longtable}
\usepackage{lscape}
\usepackage{tocloft}
\usepackage{tkz-euclide}
\usepackage{listings}
\usepackage{feynmp-auto} % Create fenynman diagrams
\usepackage{tikz-feynman} % Create fenynman diagrams
\usepackage{lipsum} % For testing. insert random text

%%%%%%%%%%%%%%%%%%%%%%%%%%%%%%%%%%%%%%%%%%%%%%%%%%%%%%%%%%%%%%%%%%%%%%%%%%%%%%%
%%% new commands and environments %%%
%%%%%%%%%%%%%%%%%%%%%%%%%%%%%%%%%%%%%%%%%%%%%%%%%%%%%%%%%%%%%%%%%%%%%%%%%%%%%%%

% Create custom font
\newenvironment{myfont}{\fontfamily{put}\selectfont}{\par}

% Adapt spacing between lines
\doublespacing

% Delete dots from toc
\renewcommand{\cftdot}{}

% Change section label to roman
\renewcommand{\thesection}{\Roman{section}}

% Customize section layout
\newcommand{\ssection}[1]{%
  \section[#1]{\centering\normalfont\scshape #1}}
\newcommand{\ssubsection}[1]{%
  \subsection[#1]{\centering\normalfont\itshape #1}}
\newcommand{\ssubsubsection}[1]{%
  \subsubsection[#1]{\centering\normalfont #1}}

% Import tikz libraries for figures
\usetikzlibrary{positioning,shadows,arrows}

% Create footnotereferencing
\makeatletter
\newcommand\footnoteref[1]{\protected@xdef\@thefnmark{\ref{#1}}\@footnotemark}
\makeatother

% Change layout of page
\hypersetup{
  colorlinks = true,
  linkbordercolor = {red},
  citebordercolor = {red},
  menubordercolor = {blue},
  urlbordercolor = {blue},
  linktoc = {page},
  pagebackref = {True},
  pdftitle = {Solution 04},
  pdfauthor = {Nils Hoyer, Maurice Morgenthaler},
  pdfcreator  = {pdflatex},
  pdfproducer = {LaTeX}
}

% Change geometry of page
\geometry{a4paper, top = 20mm, left = 20mm, right = 20mm, bottom = 15mm, headsep = 8mm, footskip = 10mm, includeheadfoot}

% Decalre uits for SIunitx
\DeclareSIUnit\femtobarn{fb^{-1}}
\DeclareSIUnit\percent{\%}

% Define colors
\definecolor{deepblue}{rgb}{0,0,0.5}
\definecolor{deepred}{rgb}{0.6,0,0}
\definecolor{deepgreen}{rgb}{0,0.6,0.2}
\definecolor{deeporange}{rgb}{0.9,0.2,0}

%%%%%%%%%%%%%%%%%%%%%%%%%%%%%%%%%%%%%%%%%%%%%%%%%%%%%%%%%%%%%%%%%%%%%%%%%%%%%%%
%%% start document %%%
%%%%%%%%%%%%%%%%%%%%%%%%%%%%%%%%%%%%%%%%%%%%%%%%%%%%%%%%%%%%%%%%%%%%%%%%%%%%%%%

\begin{document}
\begin{myfont}
\lstset{language=C++,
  basicstyle=\ttfamily,
  keywordstyle=\color{blue}\ttfamily,
  stringstyle=\color{red}\ttfamily,
  commentstyle=\color{green}\ttfamily,
  morecomment=[l][\color{magenta}]{\#}
}

\begin{center}
  \begin{Large}
    \textsc{Solution for homework assignment no. 04} \\
  \end{Large}
	\vspace*{0.4cm}
    Nils Hoyer, Maurice Morgenthaler
  \vspace*{1cm}
\end{center}

\section*{Exercise 4.1}

We are asked to write our own pseudo random number generator.
For this we will use the so called \textit{Blum Blum Shub} generator which uses the equation

\begin{equation}
n_{i+1} = n_{i}^{2} \;\%\; (p \cdot q)
\end{equation}

\noindent to generate new numbers.
$p$ and $q$ are large prime numbers. \\
To generate numbers between zero and one one has to divide by $p \cdot q$.

\begin{equation}
r_{i} = \frac{n_{i}}{p \cdot q}
\end{equation}

\noindent Please find the code in file \texttt{exercise4\_1.C}.
The 20th 'random' numbers of consecitive seeds are listed in table \ref{tab:ex_1_results}.
Just by only looking at the twenty numbers below I can not see any correlation.

{\setstretch{1.0}
\begin{longtable}{*{2}l}
  \caption[]{The last twenty random numbers are listed given twenty consecutive seeds starting at \num{234509143}.}
  \endfirsthead
  \endhead
  \toprule
  \textbf{seed $i$} & \textbf{random number $r_{i}$} \\
  \midrule
  \num{234509143} & \num{0.38945} \\
  \num{234509144} & \num{0.05298} \\
  \num{234509145} & \num{0.55513} \\
  \num{234509146} & \num{0.31014} \\
  \num{234509147} & \num{0.82852} \\
  \num{234509148} & \num{0.53909} \\
  \num{234509149} & \num{0.41962} \\
  \num{234509150} & \num{0.10040} \\
  \num{234509151} & \num{0.15332} \\
  \num{234509152} & \num{0.45729} \\
  \num{234509153} & \num{0.53053} \\
  \num{234509154} & \num{0.47494} \\
  \num{234509155} & \num{0.01497} \\
  \num{234509156} & \num{0.32598} \\
  \num{234509157} & \num{0.90585} \\
  \num{234509158} & \num{0.01488} \\
  \num{234509159} & \num{0.68329} \\
  \num{234509160} & \num{0.25334} \\
  \num{234509161} & \num{0.46313} \\
  \num{234509162} & \num{0.90314} \\
  \bottomrule
  \label{tab:ex_1_results}
\end{longtable}}

\noindent Next we used the first seed to generate \num{10000} random numbers.
We filled a histogram with them which you can see in figure \ref{fig:ex_1_results}.

\begin{figure}[H]
  \centering
  \caption{\num{10000} random number generated by our own random number generator.
  The seed which has been used is \num{234509143}.
  We used \num{200} bins for plotting in the range between zero and one. \\
  Note that the rather large differences in amounts of numbers persists even to higher total numbers of generated 'random' numbers (e.g. \num{1000000}).}
  \includegraphics[width = \textwidth]{./canvas.png}
  \label{fig:ex_1_results}
\end{figure}


\section*{Exercise 4.2}

We are asked to generate random numbers according to a specific PDF, in this case an exponential function $f(x) = e^{x}$.
In figure \ref{fig:exponential} the output of our implementation as well as the ROOT function is given. \\
As you can see there is no significant difference between both functions.

\begin{figure}[H]
  \centering
  \caption{Comparison between our own implementation and the equivalent ROOT function for \num{10000} random variables.
  You can see no significant difference between both functions.}
  \includegraphics[width = \textwidth]{./exercise4_2.png}
  \label{fig:exponential}
\end{figure}


\section*{Exercise 4.3}

We are asked to numerically verify the \textit{central limit theorem}. 
To do this we use the exponential function and compute \num{10000} averages of of $k$ random numbers.
The result can be found in figure \ref{fig:clt}. \\
You can see that increasing the total number of random numbers indeed leads to shape similar to that of a normal distribution.

\begin{figure}[H]
  \centering
  \caption{Comparison between three exponential distributions with $k$ random numbers using \num{10000} averages.
  You can see the distribution tends towards a normal distribution with an increasing number of random variables $k$.}
  \includegraphics[width = \textwidth]{./exercise4_3.png}
  \label{fig:clt}
\end{figure}



\section*{Exercise 4.4}

\begin{itemize}
  \item[\textbf{a)}] We are asked to analytically calulcate the percentage of photons hitting our detector assuming an isotropic distribution of particles. \\
  The percentage is the ratio of the surface of the detector divided by the surface of a three dimensional sphere, i.e.

  \begin{equation}
    p = \frac{O_{\textrm{segment}}}{O_{\textrm{sphere}}}
  \end{equation}

  \noindent To be able to find $p$ we need to find $r$ as only $r-h$ is given.
  From \ref{fig:ex_4_circle} you can see that

  $$
    (r -h)^{2} + a^{2} = r^{2}.
  $$

  \begin{figure}[H]
    \centering
    \caption{This figure illustrates the setup of the detector and the projection of the two dimensional surface of the sphere.
    We consider the green triangle to obtain the radius $r$ as only $r-h$ is given in the exercise.}
    \begin{tikzpicture}
      \draw[black, thick, domain=0:180] plot ({2.84*cos(\x)}, {2.84*sin(\x)});
      \draw[red, thick, domain=90:135] plot ({2.84*cos(\x)}, {2.84*sin(\x)});
      \draw (-2,2) -- (2, 2);
      \draw (0, 2) -- (0, 2.84);
      \draw (0, 0) -- (2, 2);
      \draw[dashed] (-2.84, 0) -- (2.84, 0);

      \draw[green] (0, 0) -- (0, 2);
      \draw[green] (0, 2) -- (2, 2);
      \draw[green] (0, 0) -- (2 ,2);

      \node[text width = 1cm] at (1.5, 0.85) {$r$};
      \node[text width = 1cm] at (1.3, 1.8) {$a$};
      \node[text width = 1cm] at (0.55, 2.45) {$h$};
    \end{tikzpicture}
    \label{fig:ex_4_circle}
    \end{figure}

    \noindent The segment can easily be caluclated by switching to spherical coordinates.
    You integrate the circular segment which the detector covers (indicate dby the red line in figure \ref{fig:ex_4_circle}) over $\phi$. \\
    After the integration you get 

    \begin{align*}
      O_{\textrm{segment}} & = 2\pi \cdot r^{2} \cdot \left(1 - \textrm{cos}(\theta_{0})\right) \\
                           & = 2\pi \cdot \left(a^{2} + (r-h)^{2}\right) \cdot \left( 1 - \textrm{cos}\left(\textrm{arctan}\left(\frac{a}{r-h}\right)\right)\right) \\
                           & = 2\pi \cdot \left(a^{2} + (r-h)^{2}\right) \cdot \left(1 - \frac{1}{\sqrt{\left(\frac{a}{r-h}\right)^{2} + 1}} \right)
    \end{align*}

    \noindent This results in

    \begin{equation}
      p = \frac{O_{\textrm{segment}}}{O_{\textrm{sphere}}} = \frac{2\pi \cdot \left(a^{2} + (r-h)^{2}\right) \cdot \left(1 - \frac{1}{\sqrt{\left(\frac{a}{r-h}\right)^{2} + 1}} \right)}{4\pi \cdot r^{2}} = \frac{1}{2} - \frac{1}{2\sqrt{\left(\frac{a}{r-h}\right)^{2} + 1}}
    \end{equation}

    \noindent Using $a =$ \SI{0.02}{\metre} and $r-h =$ \SI{0.10}{\metre} yields $p =$ \SI{0.971}{\percent}.



  \item[\textbf{b)}] Please find the code in the file \texttt{exercise4\_4.C}.

    \noindent The results we obtain lie around \SI{1.25}{\percent}.
    This does not match the calculated value from the previous part.    

\end{itemize}


\end{myfont}
\end{document}