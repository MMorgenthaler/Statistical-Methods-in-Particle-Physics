\documentclass[10pt]{article}

%%%%%%%%%%%%%%%%%%%%%%%%%%%%%%%%%%%%%%%%%%%%%%%%%%%%%%%%%%%%%%%%%%%%%%%%%%%%%%%
%%% packages %%%
%%%%%%%%%%%%%%%%%%%%%%%%%%%%%%%%%%%%%%%%%%%%%%%%%%%%%%%%%%%%%%%%%%%%%%%%%%%%%%%

\usepackage[english]{babel} % Choose english language
\usepackage[labelfont = bf, font = small]{caption} % Use caption package. Use bold font for caption.
\usepackage{siunitx} % Use siunitx for unit representation.
\newcommand{\RM}[1]{\MakeUppercase{\romannumeral #1{:}}}
\usepackage{graphicx}
\usepackage{tabularx}
\usepackage{float}
\usepackage{lmodern}
\usepackage{filecontents}
\usepackage{amsmath}
\usepackage{amssymb}
\usepackage[utf8]{inputenc}
\usepackage[bottom]{footmisc}
\usepackage{leftidx}
\usepackage{subcaption}
\usepackage[explicit]{titlesec}
\usepackage{booktabs}
\usepackage{multirow}
\usepackage{multicol}
\usepackage{listings}
\usepackage{enumitem}
\usepackage{pgfplots}
\usepackage{natbib}
\usepackage{xcolor}
\usepackage{url}
\usepackage{array}
\usepackage{setspace}
\usepackage{hyperref} % Referencing
\usepackage{verbatim}
\usepackage{changepage}
\usepackage[footnote, printonlyused]{acronym}
\usepackage{scrextend}
\usepackage{geometry} % Change geometry of page layout
\usepackage{rotating}
\usepackage{longtable}
\usepackage{lscape}
\usepackage{tocloft}
\usepackage{tkz-euclide}
\usepackage{listings}
\usepackage{feynmp-auto} % Create fenynman diagrams
\usepackage{tikz-feynman} % Create fenynman diagrams
\usepackage{lipsum} % For testing. insert random text

%%%%%%%%%%%%%%%%%%%%%%%%%%%%%%%%%%%%%%%%%%%%%%%%%%%%%%%%%%%%%%%%%%%%%%%%%%%%%%%
%%% new commands and environments %%%
%%%%%%%%%%%%%%%%%%%%%%%%%%%%%%%%%%%%%%%%%%%%%%%%%%%%%%%%%%%%%%%%%%%%%%%%%%%%%%%

% Create custom font
\newenvironment{myfont}{\fontfamily{put}\selectfont}{\par}

% Adapt spacing between lines
%\doublespacing

% Delete dots from toc
\renewcommand{\cftdot}{}

% Change section label to roman
\renewcommand{\thesection}{\Roman{section}}

% Customize section layout
\newcommand{\ssection}[1]{%
  \section[#1]{\centering\normalfont\scshape #1}}
\newcommand{\ssubsection}[1]{%
  \subsection[#1]{\centering\normalfont\itshape #1}}
\newcommand{\ssubsubsection}[1]{%
  \subsubsection[#1]{\centering\normalfont #1}}

% Import tikz libraries for figures
\usetikzlibrary{positioning,shadows,arrows}

% Create footnotereferencing
\makeatletter
\newcommand\footnoteref[1]{\protected@xdef\@thefnmark{\ref{#1}}\@footnotemark}
\makeatother

% Change layout of page
\hypersetup{
  colorlinks = true,
  linkbordercolor = {red},
  citebordercolor = {red},
  menubordercolor = {blue},
  urlbordercolor = {blue},
  linktoc = {page},
  pagebackref = {True},
  pdftitle = {Solution 10},
  pdfauthor = {Nils Hoyer, Maurice Morgenthaler},
  pdfcreator  = {pdflatex},
  pdfproducer = {LaTeX}
}

% Change geometry of page
\geometry{a4paper, top = 20mm, left = 20mm, right = 20mm, bottom = 15mm, headsep = 8mm, footskip = 10mm, includeheadfoot}

% Decalre uits for SIunitx
\DeclareSIUnit\femtobarn{fb^{-1}}
\DeclareSIUnit\percent{\%}

% Define colors
\definecolor{deepblue}{rgb}{0,0,0.5}
\definecolor{deepred}{rgb}{0.6,0,0}
\definecolor{deepgreen}{rgb}{0,0.6,0.2}
\definecolor{deeporange}{rgb}{0.9,0.2,0}

% Commands for further use
\newcommand{\testStat}{\tilde{q}_{\mu}}
\newcommand{\lL}[2]{\mathcal{L}\left(#1 \middle|#2 \right)}
\newcommand{\likelihood}{\mathfrak{L}}

%%%%%%%%%%%%%%%%%%%%%%%%%%%%%%%%%%%%%%%%%%%%%%%%%%%%%%%%%%%%%%%%%%%%%%%%%%%%%%%
%%% start document %%%
%%%%%%%%%%%%%%%%%%%%%%%%%%%%%%%%%%%%%%%%%%%%%%%%%%%%%%%%%%%%%%%%%%%%%%%%%%%%%%%

\begin{document}
\begin{myfont}
\lstset{language=C++,
  basicstyle=\ttfamily,
  keywordstyle=\color{blue}\ttfamily,
  stringstyle=\color{red}\ttfamily,
  commentstyle=\color{green}\ttfamily,
  morecomment=[l][\color{magenta}]{\#}
}

\begin{center}
  \begin{Large}
    \textsc{Solution for homework assignment no. 10} \\
  \end{Large}
	\vspace*{0.4cm}
    Nils Hoyer, Maurice Morgenthaler
  \vspace*{1cm}
\end{center}

\section*{Exercise 10.2}
 We were given a note on how the XENON collaboration uses a likelihood base method to set a limit on the Dark Matter - Standard Model Nuclei interaction. The following questions had to be answered.

\begin{enumerate}[label = \textbf{\roman*}.]
  \item Explain in a few words how the Profile Likelihood Ratio statistical approach works.
  \noindent
  
  \item Why is this method advantageous compared to a “hard cut in the space parameter of interest” method?  How much better is the limit with the likelihood method?
  \noindent
  
  Methods which use the "hard cut" firstly have the drawback that they are only able to give an upper limit without being able to claim a detection directly. Second, they don't take systematic uncertainties into account. Both disadvantages are not present in the presented method. ...................................How much better is the limit, tho? 

  \item Why is it important to model correctly the background and how is this done in XENON100?
  \noindent
  
  
  
  \item Which  ones  are  the  nuisance  parameters  and  how  are  they  included  in  the  full likelihood?
  \noindent
  There are five nuisance parameters which were introduced in the model.
  \begin{itemize}
      \item expected numbers of background events $N_b$
      \item the probabilities $\epsilon_s = \left\{ \epsilon^j_s\right\}$ and $\epsilon_b = \left\{ \epsilon^j_b\right\}$ 
      \item relative scintillation efficiency $\mathcal{L}_{eff}$
      \item escape velocity $v_{esc}$
  \end{itemize}
  The likelihood $\likelihood$ has the form 
  
  \begin{equation}
      \likelihood = \likelihood_1\{\sigma, N_b, \epsilon_s, \epsilon_b, \mathcal{L}_{eff}, v_{esc}; m_{\chi}\} \times \likelihood_2\{\epsilon_s\} \times \likelihood_3\{\epsilon_b\} \times \likelihood_4\{\mathcal{L}_{eff}\} \times \likelihood_5\{v_{esc}\}
  \end{equation}
  Therefore the nuisance parameters are included via the main likelihood $\likelihood_1$ as well as the subsidiary likelihoods $\likelihood_2$ to $\likelihood_5$.
  
  \item Which are the test statistics for the exclusion and the discovery case?
  \noindent
  
  In both cases the profile likelihood $\lambda{(\sigma)}$ is used.
  \begin{equation}
      \lambda(\sigma) = \frac{\max_{\sigma fixed} \likelihood\{\sigma; \mathcal{L}_{eff}, v_{esc}, N_b, \epsilon_s, \epsilon_b\}}{\max \likelihood\{\sigma; \mathcal{L}_{eff}, v_{esc}, N_b, \epsilon_s, \epsilon_b\}}
  \end{equation}
  
  For the exclusion case the test statistic $q_{\sigma}$ is defined as
  
  \begin{equation}
      q_{\sigma} = \begin{cases}
        -2 \ln{\lambda{(\sigma)}}   &   \hat{\sigma} < \sigma \\
        0                           &   \hat{\sigma} > \sigma
        \end{cases}
  \end{equation}
  
  As $0 \leq \lambda(\sigma) \leq 1$ the test statistics $q_{\sigma}$ will be $q_{\sigma} \geq 0$ in which a smaller value indicates a better compatibility between data and signal hypothesis.
  
  For the discovery case the test statistic $q_{0}$ is defined as
  
  \begin{equation}
      q_{0} = \begin{cases}
        -2 \ln{\lambda{(\sigma = 0)}}   &   \hat{\sigma} > \sigma \\
        0                           &   \hat{\sigma} < \sigma
        \end{cases}
  \end{equation}
  In this method one tries to reject the background-only hypothesis.
  
  \item What  is  the  difference  between  the  exclusion  sensitivity and  the  profiled  limit? What is the observed limit in this analysis?
  \noindent
  
  The exclusion sensitivity is the expected limit for given conditions concerning the experiment. It can be estimated before using actual data as the median of the test statistic distribution $f(q_{\sigma}|H_0)$ with background-only data simulated according to a Poisson distribution. The profiled limit on the other hand is calculated with the profile likelihood ratio method.
  The limit of this analysis is $\sigma^{up} < 2.4 \times 10^{-44} \si{cm^2}$ for WIMP's with a mass of about $m_{\chi} = 50 \si{GeV/c^2}$
\end{enumerate}


\end{myfont}
\end{document}
